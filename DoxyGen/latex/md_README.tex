\chapter{problema\+\_\+da\+\_\+8\+\_\+rainhas}
\hypertarget{md_README}{}\label{md_README}\index{problema\_da\_8\_rainhas@{problema\_da\_8\_rainhas}}
\label{md_README_autotoc_md0}%
\Hypertarget{md_README_autotoc_md0}%
O objetivo deste trabalho é utilizar o desenvolvimento orientado a testes (TDD) para verificar se um tabuleiro contém a solução para o problema das 8 rainhas. Trabalho 2 de Metados de Programação. O trabalhoo consiste em receber um arquivo .txt com um tabuleiro de xadrez, no formato de uma matriz 8x8. A matriz, os "{}1"{} representam uma rainha e os "{}0"{} representam um espaço vazio.

Para executar o código basta entrar no terminal na pasta e digitar\+:
\begin{DoxyEnumerate}
\item make -\/ para rodar o programa
\item make gcov -\/ para executar o verificador de cobertura
\item make valgrind -\/ para executar o verificador estático
\item make cpplint -\/ para ver se o código está padronizado em seu estilo
\end{DoxyEnumerate}

Para abrir a documentação Doxygren abrir o "{}html"{} o arquivo "{}index.\+html"{} 